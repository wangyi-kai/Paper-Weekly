\documentclass{article}

\usepackage{geometry}
\usepackage{graphicx}
\usepackage{subfigure}
\usepackage{diagbox}
\geometry{a4paper,left=2cm,right=2cm,top=1cm,bottom=1cm}


\title{Experiment Results}
\author{Yikai Wang}
\date{\today}

\begin{document}
\maketitle
\section{Result}
\begin{figure}[h]
\centering
\includegraphics[scale=0.5]{result1.pdf}
\includegraphics[scale=0.5]{result2.pdf}
\caption{Burgers'equation: Top:Predicted solution $u(t,x)$ along with initial and boundary training data. We are using 10,000 collocation points. Bottom:Comparison of the predicted and exact solutions corresponding to the three temporal snapshots depicted by the white vertical lines. The relative $\mathbf{L}_2$ error of this case is 2.59e-03.}
\end{figure}

\section{Training Error}
$N_u = 200, N_f = 10000$
\begin{figure}[h]
\centering
\includegraphics[scale=1]{error_compare.pdf}
\end{figure}
\section{Test error}
The architecture is fixed to 9 layers with 20 neurons per hidden layer.
\begin{table}[h]
\centering
\begin{tabular}{c|ccccc}
\hline
\diagbox{$N_u$} {$N_f$} & 2000 & 4000 & 6000 & 8000 & 10000 \\
\hline
100 & 7.8e-02 & 5.21e-02 & 5.73e-03 & 1.81e-03 & 5.03e-03\\
\hline
200 & 2.7e-01 & 3.85e-03 & 9.77e-04 & 1.13e-03 & 2.59e-03\\
\end{tabular}
\caption{Pytorch}
\end{table}

\begin{table}[h]
\centering
\begin{tabular}{c|ccccc}
\hline
\diagbox{$N_u$}{$N_f$} & 2000 & 4000 & 6000 & 8000 & 10000 \\
\hline
100 & 6.6e-02 & 2.7e-01 & 7.2e-03 & 2.2e-03 & 1.22e-03\\
\hline
200 & 1.8e-01 & 2.3e-03 & 8.2e-04 & 6.1e-04 & 5.27e-04\\
\hline
\end{tabular}
\caption{Tensorflow}
\end{table}
\end{document}