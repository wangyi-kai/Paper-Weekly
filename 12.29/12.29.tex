\documentclass{article}

\usepackage{geometry}
\usepackage{graphicx}
\usepackage{subfigure}
\usepackage{diagbox}
\usepackage{amsthm}
\newtheorem{theorem}{Theorem}[section]
\newtheorem{lemma}[theorem]{Lemma} 
\geometry{a4paper,left=2cm,right=2cm,top=1cm,bottom=1cm}

\title{Stochastic Galerkin Method}
\author{Yikai Wang}
\date{\today}

\begin{document}
\maketitle


\section{Definition of Single Random Variable}
Let $z$ be a random variable with a distribution function
$F(z)=P(z \leq Z)$, The generalized polynomial chaos basis functions are the orthogonal polynomial functions satisfying
\[\mathbf{E}[\Phi_m(z)\Phi_n(z)]=\gamma_n \delta_{nm}\]
where 
\[\gamma_n = \mathbf{E}[\Phi_{n}^{2}(z)] \]
are the normalization factors.

If $z$ is continuous, then its probability density function (PDF) exists such that $dF(z) = \rho(z)dz$ and the orthogonality can be written as
 \[ \mathbf{E}[\Phi_m(z)\Phi_n(z)] = \int \Phi_m(z)\Phi_n(z) \rho(z)dz =\gamma_n \delta_{nm}\]
Similarly, when $z$ is discrete, the orthogonality can be written as
\[\mathbf{E}[\Phi_m(z)\Phi_n(z)]=\sum_{i}\Phi_m(z_i)\Phi_n(z_i)\rho_{i} \]
\subsection{Hermite polynomial chaos}
Let $z$ is a standard Gaussian random variable with zero mean and unit variance. Its PDF is
\[\rho(z)=\frac{1}{\sqrt{2\pi}}e^{-z^{2}/2} \]
we employ the Hermite polynomials as the basis functions,
\[H_0(z)=1,\ H_1(z)=z,\ H_2(z)=z^{2}-1,\ H_3(z)=z^{3}-3z \]
This is the classical Wiener-Hermite polynomial chaos basis.
\subsection{Legendre polynomial chaos}

%The seconde Part

\section{General Procedure}
\subsection{Stochastic diffusion equation}
\[ -\nabla\cdot(a(\omega,x)\nabla u(\omega,x))=f(\omega,x)  \ (\omega,x)\in \Omega \times D\]
\[u(\omega,x)= 0 \ x \in \partial D\]
\subsection{Tensor Product}
\[ (f\otimes g)(x,y)=f(x)g(y) \]
\[ ((A\otimes B)_{i_1 j_1})_{i_2 j_2} = A_{i_1 j_1}B_{i_2 j_2} \]
\subsection{Solution}
\[\mathbf{E}[\int_{D}a\nabla u \nabla v dx]=E[\int_{D}fvdx] \]
\[v(x) = \sum_{i=1}^{N}V_i \phi_i(x) \]
\[ \int_{D}a \nabla(\sum_{i=1}^{N}U_i \phi_i(x))\cdot \nabla \phi_jdx = \int_{D}f\phi_jdx \]
\[\sum_{i=1}^{N}U_i \int_{D}a\nabla \phi_i \cdot \nabla \phi_j dx = \int_{D}f\phi_j dx \]
\[KU = F\]
\[a(\omega,x) = \bar{a}+\sum_{l=1}^{\infty}\varphi_l(x)y_l(\omega) \]
where the function $\varphi_l(x)$ are determined by the eigenvalues and eigenfunctions of the covariance function of $a(\omega,x)$, and $y_l(\omega)$ are mutually independent.
\[u(\omega,x) = \sum_{n=1}^{Q}\sum_{i=1}^{N}(U_n)_{i}\psi_n(\omega)\phi_i(x) \]
\[\mathbf{E}[\int_{D}a \nabla \sum_{i=1}^{N}\sum_{n=1}^{Q}(U_n)_i \psi_n \phi_i \cdot \nabla\psi_m \phi_j dx] = \mathbf{E}[\int_{D}f\psi_m \phi_j dx] \]
\[\sum_{i=1}^{N}\sum_{n=1}^{Q}(U_n)_i \mathbf{E}[\int_{D}a\nabla \psi_n \phi_i \cdot \nabla \psi_m \phi_j dx=\mathbf{E}[\int_{D}f\psi_m \phi_j dx] \]
\[\sum_{i=1}^{N}\sum_{n=1}^{Q}(U_n)_i \mathbf{E}[\psi_n \psi_m \int_{D}a\nabla \phi_i \cdot \nabla\phi_j dx] = \mathbf{E}[\psi_m \int_{D}f\phi_j dx] \]
\[a(\omega,x)=\sum_{l=0}^{S}a_l(x)y_l(\omega) \]
\[\sum_{l=0}^{S}\sum_{i=1}^{N}\sum_{n=1}^{Q}(U_n)_i \mathbf{E}[y_l \psi_n \psi_m \int_{D}a_l \nabla \phi_i \cdot \nabla\phi_j dx] =\mathbf{E}[\psi_m \int_{D}f\phi_j dx] \]
\[\sum_{l=0}^{S}\sum_{i=1}^{N}\sum_{n=1}^{Q}(U_n)_i \mathbf{E}[y_l \psi_n \psi_m]\int_{D}a_l \nabla \phi_i \cdot \nabla\phi_j dx =\mathbf{E}[\psi_m \int_{D}f\phi_j dx]\]
$\mathbf{E}[y_l \psi_n \psi_m]$ can pe can be evaluated
prior to any computations.
\[\sum_{l=0}^{S}\sum_{i=1}^{N}\sum_{n=1}^{Q}(U_n)_i(G_l)_{nm}(K_l)_{ij} = (F_m)_j \]
\[\sum_{l=0}^{S}\sum_{i=1}^{N}\sum_{n=1}^{Q}(U_n)_i((G_l \otimes K_l)_{nm})_{ij} = (F_m)_j \]
\[\sum_{l=0}^{S}(G_l \otimes K_l)U = F \]

\[G_{lnm} = \mathbf{E}[y_l y_n y_m \]



\[
G_{lnm}=
\left\{  
\begin{array}{lr}  
    \mathbf{E}[y]  &n = m = 0  \\
    \mathbf{E}[y]^2 &  \textbf{one of} \ n,m=0 \ \textbf{and the other} \ \neq l\\
    \mathbf{E}[y^2] & \textbf{one of} \ n,m=0 \ \textbf{and the other} \  = l  \\
    \mathbf{E}[y]^{3} & n\neq m\neq l \neq n \\
    \mathbf{E}[y^{2}]\mathbf{E}[y] & n,m,l\neq 0 \ \textbf{and exactly 2 are equal}, \\
    \mathbf{E}[y^3] & n=m=l \\
\end{array}  
\right.  
\]

\end{document}